\documentclass[14pt]{matmex-diploma-custom}
\linespread{1.5}
\usepackage{listings}
\usepackage{cite}

\usepackage{framed}
\usepackage{tikz}
\usetikzlibrary{arrows}
\usepackage{amssymb}
\newcommand{\cd}[1]{\texttt{#1}}
\usepackage[caption=false]{subfig}
\usepackage{lstcoq}
\graphicspath{{images/}}%путь к рисункам

\lstdefinelanguage{haskell}{
keywords={data, type, case, of, where, otherwise, in, let, deriving},
sensitive=true,
basicstyle=\small,
commentstyle=\scriptsize\rmfamily,
keywordstyle=\ttfamily\underbar,
identifierstyle=\ttfamily,
basewidth={0.5em,0.5em},
columns=fixed,
fontadjust=true,
literate={->}{{$\to$}}1
}

\lstset{
basicstyle=\small,
identifierstyle=\ttfamily,
keywordstyle=\bfseries,
commentstyle=\scriptsize\rmfamily,
basewidth={0.5em,0.5em},
fontadjust=true,
escapechar=~,
language=haskell
}
%\DeclareMathSizes{16}{16}{16}{16}
\definecolor{dkviolet}{RGB}{100,0,100}
\definecolor{ltblue}{RGB}{0,100,100}
\definecolor{dkblue}{RGB}{0,50,50}

\begin{document}

\filltitle{ru}{
    chair              = {Кафедра Системного Программирования},
    title              = {История язков ML},
    author             = {ИЗМЕНИТЬ ИЗМЕНИТЬ ИЗМЕНИТЬ },
    supervisorPosition = {к.\,ф.-м.\,н.},
    supervisor         = {Терехов А.\,Н.},
}
\sloppy

%\maketitle
\tableofcontents

\newpage
\section{Введение}

\setlength{\unitlength}{1mm}
\begin{picture}(160, 40)
\put(20,30){\circle{1}}А
\put(20,30){\circle{8}}
\put(20,30){\circle{16}}
\put(20,30){\circle{32}}

\put(25,10){\circle*{3}}
\put(30,10){\circle*{4}}
\put(35,10){\circle*{5}}
\put(0,40){Max Newman}
\put(10,30){Alan Turing}
\put(20,20){Christopher Stratchey}
\put(50,40){Marvyn Pragnell}
\put(40,10){Peter Landin}
\put(90,20){Rod Burstall}
\put(2,3){\oval(13,5)}
\end{picture}
\newpage
\begin{center}


\includegraphics[angle=90,scale=0.585]{Diagram.png}
\end{center}

\section{Британские исследователи языков программирования}


КАРТИНКА
Макс Ньюман, Алан Тьюринг, Кристофер Стрэтчи, Петер Лондин, Код Бёрсталл и Марвин Прагнелл
Стрэтчи был лидером  (но тут бла бла бла)

\includegraphics[angle=0,scale=0.585]{220px-Peter_Landin.png}
История про Прэгнелла и о том как он познакомился в кафе с Лондиным и книгой Principia Mathematica. Неформальная группа читателей книг. 
\begin{framed}
Бёрсталл вспоминает как спросил человека рядом в библиотеке про какую-то книгу, гно
\end{framed}

Так вот, семинар был Бёркберк колледже без разрешения руководителей колледжа. Прэгнелл нашел человека с ключом , который их впустит, где они заседали до позднего вечера. Вместе  со Стрэтчи, Лондиным и Бёрсталом туда заходил и Робин Милнер. Там они обсуждали вопросы логинки, теории категорий и кмпьютеров. Это вероятно первый раз когда они все встретились.” Так получился первый любительский undergreound семинар и компьютерах. Для рассказчика запомнилось как по-любительски всё было устроено, в отличие от того как это было сделано в штатах. И они были работающими пограммистами

\includegraphics[angle=0,scale=0.2]{stratchey.jpg}
Кристофер Стретчи (Cristopher Stratchey, 1916-1975) был другом Алана Тьюринга, когда тот работал в Кембридже и Манчестере. Он известен тем, что написал программу для игры в шашки в 1951 году в Национальной Физической Лаборатории (National Physical Laboratory, NPL). Это, возможно, была первая иградющая программа на компьютере. Эта программа запускалась на машине Марк-1 манчестерского университета, так как имела намного больше памяти чем, например, машина $ACE$, которую спроектировал Алан Тьюринг. Для Марк-1 он также написал, первую программу для проигрывания простых мелодий, а именно десткой песенки Baa Baa Black Sheep.

Что касается языков программирования, К.Стретчи спроектировал $CPL$ (Combined Programming Language), который назывался изначально Cambridge Programming language и после переезда в Лондон стал называться Combined. Также его называли Cristopher’s programming language. В этом языке были функции как сущности первого класса, а также было впервые введено понятие L-values --- выражений, которые вычисляются в место, где они хранятся. $CPL$ никогда не был успешно реализован, были прототипы и в Лондоне, и в Кембридже, но язык был очень амбициозен и сложен для того времени, так что его компиляторы никогда не появились и пользователей у него не было. Однако, Мартин Ричардс написал выпускную работу про компилятор этого языка, и вскоре решил, что было бы хорошо несколько упростить язык CPl. Так появился Basic CPL (BCPL), компилятор которого был реализован. Это был очень упешный язык для реализации больших систем в конце 60х - начале 70х, который использовался, например, в компании Xerox Park. А в Bell Labs Кен Томпсон также начал использовать его и улучшать. Так появился язык $В$, который потом эволюционировал в $С++$.

\begin{verbatim}
CPL  =>  BCPL  =>  B  =>  C  =>  C++
\end{verbatim}

Также он ввёл понятие каррирования (currying). В середине 60х Стретчи написал очень важную работу, представленную на конференции в Копенгагене в 1967, которая называлась  “Фундаментальные понятия в языках программирования “ (Fundamental concepts in programming languages). В ней он описывал различные виды полиморфизма м первым описал параметрический полиморфизм. Так же он известен как создатель денотационной семантики Скотта-Стретчи (совместно с Дана Скотт (Dana Scott) в 1969, который предоставил математическое обоснование этой семантики, основанной на лямбда-исчислении). Также Стретчи первым дал формальное определение продолжений (continuations). Он пришел к идее разделения времени (time-sharing, 1958) правда в несколько урезанном виде. В 1960-64 он нанял Питера Лондина (Peter Lundin) как своего помощника, который работал в национальной Лаборатории физики и в National Research and Develpment Corporation в Лондоне, а с 1959 Лондин стал работать как независимый консультант. 

\begin{framed}
Вот что пишет Водсворт о Стретчи:

“Стретчи имел особое чутьё в тех случаях, когда что-то было сделано “правильно” -- в основном когда было достаточно просто и достаточно элегантно, что выглядеть интуитивно правильным -- и он презирал чересчур проработанные и умудренные методы который “кое-как работали”. Его любимым принципом был такой: “Ты можешь затолкать на гору горошину носом, но это не будет являться правильным способом осуществления этого”. Я это называл “тестом Стретчи”.
”
\end{framed}
%\begin{framed}
%Бёрнсталл о Стретчи
%His elegance of Manor was accompanied by an elegance of thought and language which was a continual inspiration.
%\end{framed}

\subsection{Питер Лондин}
Питер Лондин (Peter Landin) % Фотка была выше
написал множество статей в 1960х годах.

В статье “Механическое вычисление выражений” (The mechanical evaluation of expressions, 1964) он ввел понятие SECD-машины. Это не была  первая абстрактная машина в мире, он использовал идеи сообщества программистов на языке ALGOL из города Мюнхен. 

В статье "Соответсвие между языком ALGOL60 и лямбда-нотацией Чёрча: части 1 и 2" (“A correspondence between ALGOL 60 and Chuch’s Lambda-notation: Part I and Part II”, 1965 год) он ввел понятие потоков (streams), которые являются особого рода методом для осуществления ввода-вывода.

В статье "Обобщение переходов и меток" ("A generalization of of Jumps and Labels”, 1965) он ввел понятие $J$-оператора, который является предшественником термина продолжение (continuation).

"The next 700 programming languages" 1966 года описывает ISWIM язык, который оказал влияние на ML \footnote{возможно, найти ещё про это}.

В соавторстве с Кодом Бёрнсталлом в 1969 году была опубликована статья "Programs and their Proofs: An Algebraic Approach" в которой П.Лондин предвосхитил появление алгебраических типов данных. В предыдущих работах, например про ISWIM, использовались некоторые дополнительные лингвистические конструкции для описания структур данных, которые не являлись частью языка.

В течение 1960х в MIT шла разработка Pedagogical Algorithmic Language (PAL) --- реализации ISWIM Артура Эванса. К.Стретчи приезжал пару раз в MIT в начале 60х, но дело не сдвигалось пока в 1967 году П.Лондин отправился работать в MIT, где в течение года он закончил PAL.
%, получив превратился в некоторую реализацию ISWIM. 

\subsection{Робин Милнер}

Как и К.Стретчи, и П.Лэндин, Р.Милнер не начинал как ученый -- вначале он работал школьным учителем, а перед тем как попасть в научную среду -- программистом на компанию Ferranti.
\begin{framed}
Из тьюринговской лекции:

“Идея того, чтобы машина доказывала теоремы используя логику, и идея того, чтобы используя логику понимать что машина делала... Эта двойственность вдохновляла меня в том числе и потому, что это было неочевидно”
\end{framed}

%Принципы, которыми следовали все эти люди
Кристофер Стретчи, Питер Лондин, Род Бёрнсталл, Робин Милнер (и другие, как Тони Хоар) основали британскую традицию в исследовании языков программирования и руководствовались в своей научной деятельности следующими принципами:

\begin{itemize}
 \item Осмысление важности фундаментальных оснований и семантики при изучении вычислений и программирования.
 \item Поиск ясности, строгости и элегантности путём использования математических идей и подходов, в частности из логики и алгебры.
\end{itemize}

Эти принципы особенно прижились в сообществе Эдинбурга.
 
\section{Ситуация в Эдинбурге в конце 70х}
Робин Милнер работал над LCF -- системой доказательства теорем, которая заработала в 1978-79х годах. В ней ML использовался как метаязык. %(написать ещё используя ссылку ). 
Осенью 1978 в Эдинбург приезжает Лука Карделли (Luca Cardelli) как для получения Ph.D. Род Берстолл и Дэйв МакКуин работали на языком программирования Hope -- чистым языком программирования, а также первым языком программирования с алгебраическими типами данных. Хотя за год до этого Род Берстолл сделал "игрушечный" язык программирования MPL, где абстрактные типы данных были, но в нём немного по-другому к типам добавлялись их конструкторы.  

В Эдинбурге были две исследовательские группы. Группа под руководством Рода Берстолла располагалась в Школе Искусственного Интеллекта (School of Artificial Intelligence) на какой-то там площади. Милнеровская --- в King’s Science Building на три миле южнее. Это расстояние создало существенный барьер между этими двумя группами -- не так просто было сходить туда-сюда. Решение нашлось в виде сообщества Роберта Бойера, который в те времена получал степень в Школе Искусственного Интеллекта.
%Bob Boyer memorial society. 
Когда он после защиты покинул Эдинбург, его работу как организатора начали выполнять другие люди: слушатели стали  собирался по вечерам дома у Робина Милнера  или в гостиной у кого-нибудь другого. Это сблизило два сообщества и в конце концов слушатели образовали основную часть Лаборатории Основ Информатики (Laboratory for Foundations of Computer Science, LFCS), хотя эта лаборатория формально не существовала до 1986 года.  Кроме Робина Милнера и Рода Берстолла и организационной деятельностью также занимался Мэьтю Хеннеси и Гордон Плоткин.
% Плоткин получил степень в 1972 году под руководством Бёрсталла.

В 70х Эдинбург являлся в некотором смысле центром исследования языков программирования в Европе: в этом городе работали несколько ключевых фигур, а также первое поколение их учеников. Другой центр располагался в Париже в INRIA (Institut national de recherche en informatique et en automatique, Государственный институт исследований в информатике и автоматике). Между ними были налажены прочные связи: командировки и совместные исследования спонсировались государством.

\includegraphics[angle=0,scale=0.7]{two_circles.png}

Самый первый ML, также известный как $DEC10 ML$, встраивался в систему LCF как метаязык. Логика Скотта в нём описывалась с помощью объектного языка PPLAMBDA. В ML поддерживались термы, формулы и теоремы, при этом теоремы являлись абстрактными типами и их конструирование из термов могло происходить только путём применения правил вывода, т.о. в языке не могло появиться неверных теорем. В LCF присутствовала возможность преобразования синтаксического  дерева ML в код и наоборот (техники, называемые Quotation/antiquotation). Функциональный язык программирования, был нужен, чтобы с помощью комбинирования тактик высшего порядка строить новые тактики, а сильная типизация -- чтобы не было обходных способов создать теоремы.

Основные свойства $LCF/ML$:
\begin{itemize}
  \item Основан на ICWIM.
  \item Вывод типов Через милнеровский let-polymorphism.
  \item Абстрактные типы данных (через ключевое слово abstype).
  \item Тип-сумма и тип-произведение (t1+t2, t1\#t2)
  \item Мутабельные переменные через объявление letref (с весьма нетривиальными правилами типизации)
  \item Вложенные кортежи и шаблоны для байндинга списков (pattern-matching).
  \item Условные циклы.
  \item Поддержка ошибочных ситуаций путём передачи строк, которые назывались токенами. Различные виды "ловушек" (обработчиков исключений-токенов): простые, условные и зацикливающиеся.
\end{itemize}


Компилятор для $LCF/ML$ был изначально написан на LISP (вначале на диалекте Standford, потом Rutgers). Код с $ML$ транслировался в $LISP$ и интерпретировался, поэтому работал весьма медленно. Парсер\footnote{В русскоязычной литературе также используется термин "синтаксический анализатор".} был основан на приоритетном парсере Вона Пратта (Vaugnan Pratt), который имел следующую особенность: функциии-парсеры прикреплялись к отдельно взятым символам, в то время как было бы более естественно использовать разные парсеры для одного и того же символа в зависимости от контекста. Таким образом получалось, что если запятая использовалась для разделения элементов кортежа, то её нельзя было переиспользовать для разделения элементов списка, а если точка с запятой использовалась для разделения элементов списка, то для оператора следования нужно было использовать удвоенную точку с запятой, и т.д.

\subsection{VAX ML (или Cardelli ML)}
В 1980 Лука Карделли начинает работу на собственным диалектом ML и его компилятором. К 1981 году он реализовывает его полностью на паскале, включая и сборщик мусора. В данной реализации были следующие нововведения:
\begin{itemize}	
\item Именованные структуры и варианты был основаны на плоткинских лекциях по теории доменов. До этого они не объявлялись явно перед использованием.
\item Комбинаторы для управления вычислениями (declaration combinators):
      \begin{itemize}
	\item and -- одновременно;
	\item enc -- последовательно (enclosing)  ⇒ \verb=d1;d2=;
	\item ins -- локально  ⇒ \verb=local d1 in d2 end=;
	\item rec -- рекурсивно;
	\item with -- для особой формы абстрактных типов (объявления вида \verb=with t ⇔ ty=);
      \end{itemize}
\item Оператор для ссылочного типа (ref type) вместо конструкции letref. А также правила вывода для него: ref, !, :=.
\item Ввод-вывод с помощью потоков.
\item Базовая поддержка модулей с раздельной компиляцией.
\end{itemize}

Компилятор $VAX ML$ был также реализован в Эдинбурге в 1980-1982 годах где запускался под операционной системой $VAX/VMS$. Он также был написан на паскале, включая библиотеки времени выполнения. Код компилировался в  виртуальную машину $FAM$ (Functional abstract Machine), которая генерировала код для $VAX$ машины из $FAM$ кода. В этой реализации впервые поддерживался импорт/экспорт модулей на лету. Данный компилятор вместе с операционной системой продавался пользователям начиная с 1981 года.
Экспорт/импорт модлей на лету.

Развитие $VAX ML$ показало, что ML может использоваться как язык общего назначения и имеет эффективную реализацию. Это привело к росту количества диалектов ML и последующему предложению от Робина Милнера “стандартизовать” ML (Standart ML или $SML$, апрель 1983). Таким образом,  $VAX ML$ являлся непосредственным предшественником $SML$ и тестовой площадкой для ранних экспериментов над дизайном $SML$.

Самых важные встреч для обсуждения дизайна можно выделить три: апрель 1983, июнь 1984 и май 1985. 

Так получилось, что большое количество людей собралось в Эдинбурге в апреле 1983, и Бернард Суфрин организовал встречу в гостиной Робина Милнера, где хозяин дома представил своё виденье дизайна языка\footnote{надо сслыку}. Первое черновик дизайна было написан к данной встрече и включал в себя лучшие идеи из языков $LCF/ML$, $VAX ML$ и $Hope$. 

%Физические участники
%Rod Burstall, Luca Cardelli, Guy Cousineau, Mike Gordon, David MacQueen, Robin Milner, Kevin Mitchell, Alan Mycroft, Larry Paulson, David Rydeheard, Don Sannella, John Scott, Brian Monahan, Stefan Sokolowski
%virtual participants
%Gerard Huet, Peter Mosses, David Schmidt

Название для языка Робин Милнер выбрал сам, надеясь предотвратить длительные споры о названии и дизайне. Но это оказалось ошибкой, поэтому что такое амбициозное имя в некотором смысле "приклеилось" к $ML$ и вводит в заблуждение людей мало знакомых с этой темой.

Основные возможности, заявленные в первом дизайне языка:
\begin{itemize}
\item Формы объявления типов данных, конструкторы в паттернах.
\item Без записей и вариантов (в смысле $VAX ML$)
\item Функциональные выражения с клаузами: \verb=fun v1. e1 | ... | vn. en=.
\item Мономорфные ссылка и равенство.
\item "Локальные" объявления значений вместо специального оператора \verb=ins= из $Cardelli ML$.
\item Механизм обработки ошибок (escape) с токеном и единая форма обработки (trap) для этого: \verb=e1 trap v1. e1 | ... | vn. En=. Но исключения могли в себе нести только строки.
\end{itemize}
%
%Дальнейшие черновики непосредственно языка:
%\begin{itemize}
%4/83: Changes to proposal for Standard ML, Milner
%6/83: A Proposal for Standard ML (второй черновик), Milner (49 страниц)
%11/83: A Proposal for Standard ML, Milner (27 pages) [“конечный черновки”]
%6/84: Record of the Standard ML Meeting, Edinburgh, 6-8 June 1984
%MacQueen and Milner
%7/84: Standard ML - The Core Language, Milner [changes summary]
%7/84: The Standard ML Core Language, Milner [LFP 84 draft?]
%10/84: The Standard ML Core Language, Milner
%6/85: Report on the Standard ML Meeting, Edinburgh, May 23-25, 1985, Harper
%9/85: The Standard ML Core Language (Revised), Robin Milner
%\end{itemize}

В те времена ещё не были распространены email и ARPANET, поэтому вся переписка велась через почтовое сообщение. Постепенно спецификация стандарта менялась, наращивая небольшие изменения. В июне 1985 года люди собрались во второй раз, называв эту встречу ML Workshop. Так состоялся первый ML Workshop который проводится каждый год по сей день.

По мимо черновиков стандарта самого языка рассматривались предложения по системам ввода-вывода и модулей.

%Other Design Drafts - I/O and Modules
%Stream I/O:
%12/83: Stream Input/Output, Cardelli [Polymorphism 3,1]
%2/85: Proposal for I/O in Standard ML, K. Mitchell and Milner
%6/85: Standard ML Input/Output, Harper [ML Workshop 85]
%Modules:
%8/83: Modules for Standard ML, MacQueen [preliminary, incomplete draft]
%8/84: Modules for Standard ML, MacQueen [LFP 84, Polymorphism]
%10/85: Modules for Standard ML, MacQueen [final draft before Definition]




\subsection{Standard ML (SML 90)}
Работы над формальным описанием языка начались в районе 1986 года. Всего было в Эдинбурге было выпущено три издания стандарта языка. Работы велись преимущественно Р.Милнером, Робертом Харпером и учеником Милнера Мэдсом Тофтэ.
\begin{itemize}
\item 8/87: The Semantics of Standard ML, издание 1
\item 8/88: The Definition of Standard ML, издание 2
\item 5/89: The Definition of Standard ML, издание 3
\end{itemize}
Финальная версия спецификации языка была выпущена в 1990 году в издательстве MIT Press.

Одно из самых важных нововведений $SML 90$, которое было предложено не в самом начале, была поддержка исключений как расширяемых типов данных: язык был расширен конструкторам исключений и паттерн-матчингом исключений в их обработчике. 

Три ранние реализации Standart ML:
\begin{itemize}
\item Cardelli’s VAX ML можно назвать "полустандартным" ML -- в нём были поддержанные особенности языка, предложенные в 1983-84 годах.
\item Edinburgh ML => Edinburgh SML.
\item Кэвин Митчелл (Kevin Mitchell), Алан Майкрофт (Alan Mycroft), Джон Скотт (John Scott) и Боб Харпер (Bob Harper) переписал на  Cardelli’s ML компилятор Cardelli ML.
\item Дэйва Мэтьюс (Dave Matthews) реализовал на SML фронтенд для своего языка PolyML.
\end{itemize}

Более поздние реализации:
\begin{itemize}
 \item Standart ML of New Jersey
 \item MLKit (with Regions)
 \item Moscow ML
 \item MLton
\end{itemize}

\subsection{Главные идеи в  Standard ML ‘90}
\begin{itemize}
\item Let-полиморфизм, вывод типов, наиболее общие (principal) типы: статьи Ньюмана (1943!), Хаскелла Карри (1969), Рождера Хиндли (1969), Роберта Милнера.
\item Алгебраические типы данных и функции с клаузами, разбор случаев с помощью паттерн-матчинга (заимстовано из языка $Hope$).
\item Модули и их сигнатуры, функторы, спецификации разделения данных и генеративные структуры (“strong structure sharing”): структуры  с  некоторой статической уникальной идентичностью, которые  можно было сравнивать, т.е. на этапе компиляции проверять, что структуры равны. У этой особенности языка получилась очень сложная семантика, потому она не стало особо популярной.
\item Исключения как расширяемый тип данных.
\item Поддержка изменяемых значений (а именно ссылочных типов -- ref types) используя  понятие "императивных переменных типа" (imperative type variables) привела к некоторому количеству сложностей в языке. Было написано большое количество статей на тему того как улучшить данных подход и почему от него стоит отказаться.
\end{itemize}

Эволюция алгебраических типов данных:
\begin{enumerate}
\item Неформальные описания данных, использованные в ISWIM.
\item Формальное развитие этих методов в работе Люндина и Бёрсталла “Programs and Their Proofs: An Algebraic Approach” (полуформально описаны).
\item “Игрушечный” язык Бёрсталла NPL.
\item В Языке HOPE более-менее полно софрмировались и перекочевали оттуда в Standart ML.
\end{enumerate}



%Картинка
%\begin{verbatim}
%datatype AE = ID of identifier
%            | LAMBDA of {bv: identifier, body : AE}
%            | COMB of {rator : AE, rand : AE}
%\end{verbatim}

\subsection{Ошибки в процессе дизайна}
"Замораживание” формального описания языка в виде книги в виде книге оказалось плохой идеей. Если язык программирования обретает реализацию и начинает использоваться, то его спецификацию необходимо поддерживать, а также (весьма осторожно) позволять языку изменяться. Описание должно быть открытым, но при этом очень аккуратно поддерживаемым.
Возникновение этой проблемы связано со степенью участия Робина  Милнера в проекте. Он был весьма занятым человеком, его основными проектами в то время были CCS и пи-исчисление. Ему хотелось дойти до состояния, когда он может передать рукопись в издательство MIT  и перестать думать об этом проекте. Для своих целей он не планировал реализовывать поддерживать и даже использовать этот язык. Следует помнить, что в те времена не существовало WWW, так что не было варианта выложить описание в онлайн.

Выпущенная книга про SML’90 имела очень короткое (2-3 страницы) приложение про “основы”, что по сути являлось стандартной библиотекой типов и функций. Список был очень короткий и состоял только из 43х пунктов, таким образом, каждый кто реализовывал $SML$ должен был расширять этот перечень, и каждая реализация расширяла этот список несовместимым друг с другом способом. Процесс "устаканивания" занял довольно длительное время, до тех пор пока Джон Реппи (John Reppy) не взялся за эту проблему и не создали “SML Basis Library”. Однако, эта работа вышла в свет лишь спустя несколько лет после SML‘97.

% там кроме Reppy был Gasner но лень искать кто это.

\subsection{Развитие языка в 1990е годы}
В 1997 году институт Ньютона (Isaac Newton Institute for Mathematical Sciences), а именно программа по исследованию семантики вычислений собрала Р.Милнера, Харпера, М.Тофтэ и Д.МакКуина в Кембридже, где они начали работу по переработке описания языка Standart ML. Вот значимые изменения:

\begin{itemize}
\item Сокращения типов в сигнатурах (SML/NJ 0.93; Harper, Leroy POPL 94).
\item Непрозрачное сравнение сигнатур.
\item Слабое разделение сигнатур (подразумевает собой только равенство типов).
\item Полиморфизм значений: уход от императивных типовых переменных, переход к ограничению полиморфизма значений (value restriction).
\item Клонирование типов данных.
\end{itemize}
%Также хотели выкинуть equality types, но Милнер застеснялся.

\subsection{ML2000}
Серия встреч между 1993 и 2000 годами была посвящена идее создания ML “следующего поколения”, но консенсус так и не был пройден в основном из-за не согласия участников по поводу добавления объектно-ориентированных возможностей в язык. В то время уже существовали языки со своей собственной реализацией объектов такие как OCaml, а также язык Moby за авторством Reppy и Кэтлин Фишер (Kathleen Fisher). 
Статья Principles and Preliminary Design of ML2000. 


\section{A History of OCaml}
Изначально, “Caml” являлся акронимом для Categorical Abstract Machine Language -- языком для Категориальной Абстрактной Машины. Имя языка является слиянием названия машины и семейства языков, которому он принадлежит. Название $Caml$ сохранилось на протяжении всей эволюции языка, не смотря на то, что современная реализация не имеет никакого отношения к $CAM$.

Caml был создан в исследовательской группой Formel из INRIA, которой руководил Жерар Уэ (Gérard Huet). Его развитие продолжалось в исследовательских группах Cristal, и современной Gallium.

\subsection{Истоки}
Лаборатория Formel начала интересовать языком ML в начале 80х годов. В те времена ML использовался в LCF, а сам LCF был написан частично на LISP и частично на ML. В Formel использовались несколько операционных систем (Multics, Berkeley Unix on Vax, Symbolics), поэтому Жерар Уэ решил создать реализацию ML, которая будет совместима одновременно с несколькими компиляторами LISP  (MacLisp, FranzLisp, LeLisp, ZetaLisp). В работу были вовлечены Гай Кузино (Guy Cousineau) и Ларри Паульсон (Larry Paulson). В придачу к созданию компилятора была существенна увеличена производительность языка.

Гай Кузино, следуя идеям Робина Милнера, также добавил в язык pattern-matching и алгебраические типы данных, которые он заимствовал из языка Hope, спроектированного Родом Берстоллом и Дэйвом МакКуином. Некоторое время эта реализация называлась "Le ML", но это названия не прижилось. It was used by Larry Paulson to develop Cambridge LCF and by Mike Gordon for the first version of HOL, as recalled in Gordon's short history of HOL.

Около 1984 года, когда Робин Милнер предложил формальную спецификацию ML, а Лука Карделли уже имел эффективную реализацию ML для FAM, Пьер-Луи Курье разработал исчисление категориальных комбинаторов, а также соответствие между ними и лямбда-исчислением. Это наблюдение использовал Гай Кузино для реализации эффективного способа компиляции в ML. Этот способ был близок к тому, что был использован в $Edinburgh ML$, но мог быть формально описан, доказан и по-простому оптимизирован. Это привело к появлению Категориальной Абстрактной Машины. 

Таким образом Гаю Кузино была необходима новая реализация, основанная на CAM. Однако, в итоге получился не Standart ML, а Caml. Почему? Главной причиной для разработки Caml было то, что он использовался в лаборатории Formel как основной язык для внутренней разработки. Также он использовался для разработки системы $Coq$, которая после появления выпускной работы Тьери Кокана (Thierry Coquand) стала основным направлений исследований в Formel. Группа не приняля  "стандартного" синтаксиса, потому что он мог помешать адаптации языка под текущие нужды. В частности, Филипп Лё Шенадэ (Philippe Le Chenadec) и Мишель Мони (Michel Mauny) разработали утилиты для манипуляции синтаксисом, которые оказались весьма полезными и были интегрированы в Caml. Обсуждение изменений, которые выглядели полезными, с разработчиками $Standart ML$ внесла бы чересчур большую задержку в работу. Более того, философия ведения разработки противоречила "стандартному" языку, где не предполагалось слишком быстрых изменений. Таким образом, все существенные улучшения, которые появлялись в $Standart ML$ и $Edinburgh ML$ вручную переносились в Caml.

\subsection{Первая реализация}
Первая реализация Caml появилась в 1987 году и продолжала своё развитие до 1992 года. Она были сделана в основном Аскандером Суарезом (Ascander Suarez). После того, как он покинул лабораторию в 1988 году, Пьер Вей (Pierre Weis) и Мишель Мони (Michel Mauny) занимались развитием и поддержкой до 1992 года. Эта реализация компилировала Caml в  $LLM3$, виртуальную машину системы $Le\ Lisp$. Гай Кузино скромно вспоминает: "Я должен признать, что когда начиналась разработка Caml, я имел весьма небольшой опыт в языках программирования. То, что мы взяли за основу систему $Дe Lisp$ и её механизмы по выделению памяти и сборке мусора, сэкономило нам много времени, но не позволяло получить высокой производительности. Модель CAM позволяла быстро конструировать замыкания и разделять области видимости, но с доступом к окружению и применением оптимизаций были сложности. Также она могла порождать утечки памяти, так как ненужные значения сохранялись в замыканиях. Также, я не сразу осознал, что более важным является хорошая производительность не-функциональных программ, чем полностью  написанных в функциональном стиле. Также я недооценил важности портируемости языка на другие платформы и открытости модели разработки. Не смотря на эти неувязки, ответственность за которых я беру на себя, Аскандер, Пьер и Мишель выполнили великолепную работу."


\subsection{Caml Light}
В 1990 и 1991 годах Завье Леруа (Xavier Leroy) спроектировал абсолютно новую реализацию Caml, в основе которой лежал интерпретатор байт когда, написанный на С. А Дамьен Долигэ (Damien Doligez) спроектировал великолепную систему управления памятью. Эта новая реализация, получившая название $Caml Light$, была легко портируемой и запускалась на таких небольших домашних компьютерах как Mac'и и PC. Она заменила старую реализацию Caml и привела к росту популярности Caml в системах высшего образования и научно-исследовательских лабораториях. Поддержка потоков данных и технологии синтаксического анализа (благодаря Мишелю Мони) позволила команде Formel создать необычных системы управления синтаксисом. 

\subsection{Caml Special Light}
В 1995 году Завье Леруа анонсировал $Caml Special Light$, который улучшил $Caml Light$ в нескольких аспектах. Во-первых, это оптимизирующий компилятор в машинный код, который был сравним или превосходил производительность лучших на тот момент компиляторов функциональных языков и позволил Caml соревноваться в производительности с с такими языками как С++. Во-вторых, в $Caml Special Light$ была высокоуровневая система модулей, спроектированная Завье Лероем на основе системы модулей в SML. Эта система предлагала богатые выозможности абстракции и программирования в целом. 
% "в целом"?

\subsection{Objective Caml}
Системы типов и системы вывода типов для объектно-ориентированных языков были популярной темой для исследования в начале 1990х годов. Дидье Рэми (Didier Rémy), к которому позднее присоединился Жером Вульон (Jérôme Vouillon), разработал элегантную и выразительную систему типов для объектов и классов. Реализация была интегрирована в $Caml Special Light$, что привело к появлению в 1996 году реализации языка Objective Caml, который был переименован в OCaml в 2011 году. Он является первым языком, который сочетает выразительность объектно-ориентированного программирования со статической типизацией в стиле ML и выводом типов.  Он поддерживает большинство современных идиом объектно-ориентированного программирования (параметрические класс, бинарные методы, специализация) статически и типобезопасно, в то время как в других языках (С++ и Java) это приводит к противоречивости системы типов или необходимости проверок во время выполнения.

В 2000 году Жак Гариг (Jacques Garrigue) расширил $Objective Caml$ некоторыми возможностями, над которыми экспериментировал в диалекте $Objective Label$. Среди них полиморфные методы, именованные и опциональные параметры функций, полиморфные варианты.
% полиморфные методы ?????
%In 2000, Jacques Garrigue extended Objective Caml with several new features, which he had been experimenting with for a few years in the Objective Label dialect of Objective Caml. Among these features were polymorphic methods, labeled and optional function arguments, and polymorphic variants.

Расцвет OCaml пришелся на конец 1990х, когда он начал набирать популярность и привлекать существенное количество пользователей.
%The rise of OCaml
%Since the late 1990's, OCaml has been steadily gaining in popularity and attracted a significant user base. In addition to impressive programs developed in %OCaml, the user community also contributed many high-quality libraries, frameworks and tools in areas ranging from graphical user interfaces and database %bindings to Web and network programming, cross-language interoperability and static program analysis. In parallel, the core OCaml development team actively %maintains the base system, improving the quality of the implementation and porting it to the latest architectures and systems. As lead developer of OCaml, %Chair of the Caml Consortium and Owner of OCaml.org, Xavier is considered to be benevolent dictator for life (BDFL) of the OCaml language.

%\subsection{Некоторые ближайшие родственники}
%In addition to these mainstream versions of Caml, one should mention many related compilers. Michel Mauny and Daniel de Rauglaudre designed Chamau, which %offers unique syntax manipulation facilities which are now offered in the Camlp4 pre-processor for OCaml.
%Manuel Serrano and Pierre Weis created BIGLOO. Régis Cridlig made Camlot. Jean Goubault-Larrecq wrote HimML, which features implicit hash-consing and %efficient operations on sets and maps. Emmanuel Chailloux published CeML. In the Para team, Francis Dupont implemented Caml for parallel machines, while %Luc Maranget built Gaml, a compiler for a lazy functional programming language.

В 1996 году Гай кузино писал: "Конечно, история Caml могла быть более линейной. Однако, методом проб и ошибок, во Франции появился функциональный язык программирования достаточно гибкий, портируемый и обладающий хорошей производительностью."

\newpage
\begin{thebibliography}{9}
  \bibitem{hilGeo}
    Д. Гильберт. Основания геометрии, перевод с немецкого под редакцией А.В.Васильева,
    Л., "Сеятель", 1923 — 152 с.
  \bibitem{kushner}
    Б.А. Кушнер. Лекции по конструктивному математическому анализу. — М.: Наука, 1973. — 447 с.
  \bibitem{taran}
    К. Таран. Метод применения Теории Типов Мартина-Лёфа для верификации программных систем.
    Дипломная работа, кафедра СП, СПбГУ, 2014.
  
  \bibitem{baren}
    H. Barendregt. Lambda Calculi with Types, Handbook of Logic in Computer Science, Volume II, Oxford University Press, 1991.
  \bibitem{coc}
    T. Coquand, G. Huet. The Calculus of Constructions, Information and Computation 76 (2–3), 1988.
  \bibitem{curry}
    H.B. Curry, R. Feys. Craig, William, ed., Combinatory Logic Vol. I, Amsterdam: North-Holland, 1958. p. 9E.
  \bibitem{gont}
    G. Georges. Formal Proof—The Four-Color Theorem, Notices of the American Mathematical Society 55 (11): 1382–1393, 2008.
  \bibitem{Hott}
    Homotopy Type Theory: Univalent Foundations of Mathematics. — Princeton: Institute for Advanced Study, 2013.
  \bibitem{howard}
    W.A. Howard. The formulae-as-types notion of construction, in Seldin, Jonathan P.; Hindley, J. Roger, To H.B. Curry: Essays on Combinatory Logic, Lambda Calculus and Formalism, Boston, MA: Academic Press, 1980 (original paper 1969). pp. 479–490
  \bibitem{makkai}
 M. Makkai. First Order Logic with Dependent Sorts, with Applications to Category Theory, 1995.
  \bibitem{progML}
    B. Nordström, K. Petersson, J. M. Smith. Programming in Martin-Löf's Type Theory. Oxford University Press, 1990.
  \bibitem{norell}
    U. Norell. Towards a practical programming language based on dependent type theory. PhD Thesis. Chalmers University of Technology, 2007.
  \bibitem{itt} P. Martin-Löf. Intuitionistic type theory,
    Studies in proof theory: Lecture notes (1), Giovanni Sambin, Bibliopolis, 1984.
  \bibitem{presburger}
    M. Presburger. Über die Vollständigkeit eines gewissen Systems der Arithmetik ganzer Zahlen, in welchem die Addition als einzige Operation hervortritt. Comptes Rendus du I congrès de Mathématiciens des Pays Slaves, 1929.  Warszawa. p. 92-101.
  \bibitem{russel1}
    A.N. Whitehead, B. Russell Principia mathematica 1 (1 ed.), Cambridge: Cambridge University Press, 1910.
  \bibitem{russel2}
    A.N. Whitehead, B. Russell. Principia mathematica 2 (1 ed.), Cambridge: Cambridge University Press, 1912.
  \bibitem{russel3}
    A.N. Whitehead, B. Russell. Principia mathematica 3 (1 ed.), Cambridge: Cambridge University Press, 1915.
  \bibitem{zhao}
     L. Zhaohui. Computation and reasoning: a type theory for computer science. Oxford University Press, Inc., 1994.
\end{thebibliography}

\end{document}
